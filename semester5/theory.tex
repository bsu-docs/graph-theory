\documentclass[a4paper,openany]{book}
\usepackage[utf8]{inputenc}
\usepackage[russian]{babel}
\usepackage{mathtools}
\usepackage{amssymb}
\usepackage{amsthm}
\usepackage{tikz}
\usepackage{multicol}
\usepackage{xparse}
\usepackage{enumitem}
\usepackage{centernot}
\usepackage{comment}

\usepackage[left=1cm,right=1cm,top=1cm,bottom=1cm]{geometry}

% \usepackage{setspace}
%\doublespace
\usepackage{epstopdf}
\usepackage{graphicx}
\usepackage{titlesec}
\newcounter{TheoremCounter}
\newtheorem{theorem}[TheoremCounter]{Теорема}
\newtheorem*{lemma}{Лемма}

\newcommand{\norma}{}
\usepackage{mainstyle}
\renewcommand{\theenumii}{\asbuk{enumii}}

% For cyrillic symbols in "enumerate" environment.
\renewcommand{\theenumii}{\asbuk{enumii}}
\AddEnumerateCounter{\asbuk}{\@asbuk}{ы}


%====================================================================================
%   ENVIORONMENTS

\newenvironment{definition}
{\begin{statement}{Определение}}
    {\end{statement}}

\newenvironment{characteristics}
{\begin{statementItemed}{Свойства}}
    {\end{statementItemed}}

\newenvironment{note}
{\begin{statementDotted}{Замечание}}
    {\end{statementDotted}}

\newenvironment{notes}
{\begin{statementItemed}{Замечания}}
    {\end{statementItemed}}

\newenvironment{proofUndotted}
{{\raggedright \textit{Доказательство}}}
{\begin{flushright}
       \boxed{}
 \end{flushright}
}

\newenvironment{noteo}{}{}
\RenewDocumentEnvironment{noteo}{o}
{{\raggedright \textbf{Замечание}\IfValueTF{#1}{ (\textit{#1})}{}.}$  $

}
{}

\newenvironment{consequence}{}{}
\RenewDocumentEnvironment{consequence}{o}
{{\raggedright \textbf{Следствие}\IfValueTF{#1}{ (\textit{#1})}{}.}$  $

    }
{}

\NewDocumentEnvironment{consequences}{o}
{\raggedright \textbf{Следствия}\IfValueTF{#1}{(\textit{#1})}{}:\begin{enumerate}}
    {\end{enumerate}}


% \RenewDocumentEnvironment{theorem}{o}
% {{\raggedright
%   \textbf{Теорема}\IfValueTF{#1}{ (\textit{#1})}{}}.$  $
%
% }

\NewDocumentEnvironment{stm}{o}
{{\raggedright
  \textbf{Утверждение}\IfValueTF{#1}{ (\textit{#1})}{}}.$  $

}

\NewDocumentEnvironment{plan}{o}
{\raggedright \textbf{Краткий план}\IfValueTF{#1}{(\textit{#1})}{}:\begin{enumerate}}
	{\end{enumerate}}

\newenvironment{theoremNamed}{}{}
\RenewDocumentEnvironment{theoremNamed}{m o}
{{\raggedright
  \textbf{Теорема #1}\IfValueTF{#2}{ (\textit{#2})}{}}.$  $

}


\RenewDocumentEnvironment{lemma}{o}
{{\raggedright
        \textbf{Лемма}\IfValueTF{#1}{ (\textit{#1})}{}}.$  $

}

\newenvironment{lemmaNamed}{}{}
\RenewDocumentEnvironment{lemmaNamed}{m o}
{{\raggedright
        \textbf{Лемма #1}\IfValueTF{#2}{ (\textit{#2})}{}}.$  $

}


\newenvironment{exercise}
{\begin{statementDotted}{Упражнение}}
    {\end{statementDotted}}

\newenvironment{example}
{\begin{statementDotted}{Пример}}
    {\end{statementDotted}}

\newenvironment{examples}
{\begin{statementItemed}{Примеры}}
    {\end{statementItemed}}

% =================================
% math functions
\newcommand{\abs}[1]{
    \left\lvert #1 \right\rvert
}

\newcommand{\maxf}[1]{
    \max \left\{ #1 \right\}
}

\newcommand{\suchthat}{
    \;\ifnum\currentgrouptype=16 \middle\fi|\;
}

\newcommand{\arc}[1]{
    \buildrel\,\,\frown\over{#1}
}

\DeclareMathOperator{\const}{\text{const}}

\DeclareMathOperator{\fix}{\text{fix }}
\DeclareMathOperator{\diam}{diam\,}
\DeclareMathOperator{\mes}{mes}
\DeclareMathOperator{\divergence}{div}

\DeclareMathOperator{\arcsh}{arcsh}
\DeclareMathOperator{\arth}{{arth}}
\DeclareMathOperator{\arcth}{{arcth}}
\DeclareMathOperator{\sgn}{sgn}

% \DeclareMathOperator{\deg}{deg}

\newcommand{\parenthesis}[1]{%
    \left( #1 \right)
}

\newcommand{\plot}[1]{$ \text{Г}_{#1} $}

\newcommand{\limlim}[2]
{ \lim\limits_{ \substack{ #1 \\ #2 } } }

\newcommand{\limitslimits}[2]
{ \limits_{ \substack{ #1 \\ #2 } } }

\newcommand{\limlimlim}[3]
{ \lim\limits_{ \substack{ #1 \\ #2 \\ #3 } } }

\newcommand{\nullFrac}{\dfrac{ }{}}

\renewcommand{\emptyset}{\varnothing}

\newcommand{\diint}{\displaystyle\iint}

\newcommand{\liml}{\lim\limits}
\newcommand{\intl}{\int\limits}
\newcommand{\iintl}{\iint\limits}
\newcommand{\diintl}{\diint\limits}
\newcommand{\dintl}{\dint\limits}

\newcommand{\suml}{\sum\limits}
\newcommand{\sumnzi}{\sum\limits_{n=0}^{\infty}}
\newcommand{\limninf}{\lim\limits_{n\to\infty}}
\newcommand{\dintlzi}{\dintl_0^{+\infty}}


%========================================================
% text style

\newcommand{\important}[1]{\textit{#1}}

\newcommand{\dint}{\displaystyle\int}
\newcommand{\dsum}{\displaystyle\sum}

\newcommand{\oiint}[2]{
    \begin{tikzpicture}[baseline=(C.base)]
        \node(C) {$ \displaystyle \iintl_{#1}^{#2} $};
        \draw (0,0.15) circle (0.25);
        %\node[draw,circle,inner sep=1pt](C) ++ (0, 0.1) {$ \;\;\;\; $};
    \end{tikzpicture}
}

\newcommand{\circled}[1]{
    \begin{tikzpicture}[baseline=(C.base)]
    \node[draw,circle,inner sep=1pt](C) {#1};
    \end{tikzpicture}
}

\newcommand{\eqlhopital}{%
    \overset{\circled{Л}}{=}}

\newcommand{\neqlhopital}{%
    \overset{\circled{Л}}{\neq}}

\newcommand{\sqcase}[1]{%
    \left[\begin{matrix}#1\end{matrix}\right]
}

\newcommand{\Arg}{%
  \operatorname{Arg}
}

\newcommand{\Ln}{%
  \operatorname{Ln}
}

\newcommand{\Arsh}{%
  \operatorname{Arsh}
}

\newcommand{\Arch}{%
  \operatorname{Arch}
}

\newcommand{\Arth}{%
  \operatorname{Arth}
}

\newcommand{\Arcth}{%
  \operatorname{Arcth}
}

\newcommand{\Arcsin}{%
  \operatorname{Arcsin}
}

\newcommand{\Arccos}{%
  \operatorname{Arccos}
}

\newcommand{\Arctg}{%
  \operatorname{Arctg}
}

\newcommand{\Arcctg}{%
  \operatorname{Arcctg}
}

\renewcommand{\th}{%
  \operatorname{th}
}

\renewcommand{\cth}{%
  \operatorname{cth}
}


\newcommand{\dvert}{\left.\nullFrac\right\vert}

\renewcommand{\r}[1]{$\overset{\text{ }_\bullet\text{ }}{\text{#1}}$}

\renewcommand{\norma}[1]{\left\lvert\left\lvert#1\right\rvert\right\rvert}
\newcommand{\norm}[1]{\left\lvert\left\lvert#1\right\rvert\right\rvert}

\newcommand{\RN}{\mathbb{R}^n}
\newcommand{\R}[1]{\mathbb{R}^{#1}}

% =================================
% sets utilities

\newcommand{\defineset}[2]{
    \left\{ #1 \, \middle\vert \, #2 \right\}
}

\newcommand{\set}[1]{
    \left\{ #1 \right\}
}

\newcommand{\colquestion}[1]{\section{#1}}
\newenvironment{col-answer-preambule}
               {\ignorespaces}
               {\ignorespacesafterend}

\begin{document}
\begin{center}
  \LARGE\underline{\textbf{Теория графов}}\\
\end{center}

Конспект:
https://drive.google.com/open?id=0B7NhKjbM4DhuVkNYenBQc2NzZEU

Нам не надо доказательства знать

План:
0. Введение
1. Деревья
2. Связность
3. Независимость и покрытие
4. Планарность
5. Обходы в графах
6. Раскраски
7. Ориентированные графы
8. Степенные последовательности
9. Гиперграфы
10. Алгоритмические аспекты теории графов. Теория сложности



\chapter{Введение}

\begin{stm}
  Объединение двух несовпадающих (a,b)-цепей содержит простой цикл.
\end{stm}
\begin{stm}
  Пусть C1и C2- два различных простых цикла, у которых имеется общее ребро e, тогда (С1С2)-e также содержит простой цикл.
\end{stm}
\begin{stm}
  Для связности графа необходимо и достаточно, чтобы в нем для какой-либо фиксированной вершины u и каждой другой вершины v существовал (u,v)-маршрут.
\end{stm}
\begin{stm}
  Каждый граф представляется в виде дизъюнктого объединения своих связных компонент ( однозначным образом).
\end{stm}
\begin{stm}
  G-связный граф, e-ребро, если e принадлежит циклу, то граф G-e связен.
\end{stm}
\begin{stm}
  G-связный граф, если e-ребро не принадлежит циклу, то G-e имеет 2 компоненты связности.
\end{stm}

\begin{definition}
  Мультиграф - допускаются кратные рёбра, псевдограф - допускаются петли
\end{definition}

\chapter{Деревья}
\begin{definition}
  Граф будем называть ацикличным, если в нём нет циклов.
\end{definition}
\begin{definition}
  (n,m)-графом будем называть граф, содержащий n вершин и m рёбер.
\end{definition}

\begin{theorem}
  Если граф $G - (n,m)$-граф, тогда следующие утверждения эквивалентны:
  \begin{enumerate}
    \item $G$ - дерево
    \item $G$ - связен и $m=n-1$
    \item $G$ - ацикличен и $m=n-1$
    \item $G$ - ацикличен и при соединении любой пары несмежных вершин ребром
    образуется ровно 1 цикл.
  \end{enumerate}
\end{theorem}

\begin{definition}
  Матрица Кирхгофа $B(G)$:
  \begin{itemize}
    \item $B_{ij} = 0$, $i$ и $j$ не смежны
    \item $B_{ij} = -1$, $i$ и $j$ смежны
    \item $B_{ij} = \deg(i), i = j$
  \end{itemize}
\end{definition}

\begin{theorem}[Киргхгофа]
  Число остовных деревьев в графе G, $|G| = n >= 2$, если G - связен, равно
  алгебраическому дополнению любого элемента матрицы Кирхгофа.
\end{theorem}

\begin{consequence}
  Число помеченных деревьев на n вершинах равно $n^{n-2}$.
\end{consequence}
\chapter{Связность}
\begin{definition}
  Наименьшее число вершин удаление которых приводит к несвязности графа (или одной вершине) называется числом вершинной связности $\kappa(G)$ .
\end{definition}
\begin{definition}
  Реберной - аналогично. Обозначается $\lambda(G)$
\end{definition}
\begin{definition}
  Вершина $v$ такая, что $G-v$ имеет больше компонент связности, чем $G$, называется точкой сочленения.
\end{definition}
\begin{definition}
  Ребро $e$, такое, что $G-e$ имеет больше компонент связности, чем G, называется мостом.
\end{definition}

\begin{theorem}
  Для любого графа справедливо $\kappa(G) \leqslant \lambda(G) \leqslant \delta(G) $.
\end{theorem}

\begin{definition}
  Граф называется k-связным, если $\kappa(G)$k, k-реберно связным, если (G)k.
\end{definition}

\begin{definition}
  k-связной компонентой называется максимальный k-связный подграф.
\end{definition}

\begin{note}
  Определение связности, 2-связности, 3-связности можно реализовать за линию. А дальше темный лес без полиномиальных алгоритмов. Такие дела.
\end{note}

\begin{theorem}
  Две различные $k$-компоненты графа имеют не более чем $(k-1)$ общих вершин.
\end{theorem}

Немного неопровержимых утверждений:
\begin{enumerate}
  \item Степени вершин 2-связного графа $\geqslant 2$.
  \item Если $G_1$ и $G_2$ - двусвязные и имеют не менее двух общих вершин, то
  их объединение тоже двусвязный граф.
  \item Если граф 2-связный и P-простая цепь, то $G \cup P$ -2-связный
  \item Если вершина $v$ не точка сочленения связного графа, то любые две его
  вершины можно соединить цепью, не проходящей через v.
  \item Для любых трех несовпадающих вершин 2-связного графа существует простая цепь,
  которая соединяет две и не проходит через третью.
\end{enumerate}

\begin{theorem}
  G-связный, $|G| > 2$, тогда следующие утверждения эквивалентны.
  \begin{enumerate}
    \item G - 2-связный.
    \item Любые две вершины принадлежат общему простому циклу
    \item Любая вершина и любое ребро принадлежат общей простой цепи.
    \item Любые два ребра принадлежат общей простой цепи.
    \item Для любых вершин $a$ и $b$ и ребра $е$ существует простая $(a,b)$-цепь, которая содержит это ребро.
    \item Для любых вершин $a,b,c$ существует простая $(a,b)$-цепь, которая проходит через $с$.
  \end{enumerate}
\end{theorem}

\begin{definition}
  Максимальный подграф графа G, который не имеет собственных точек сочленения называется блоком.
\end{definition}

Свойства блоков:
\begin{enumerate}
  \item Любые два блока графа имеют не более одной общей вершины.
  \item Если блок содержит вершины x, y, то он содержит и всякую простую (x,y)-цепь этого графа
  \item Если вершина входит в более, чем один блок, то она - точка сочленения.
\end{enumerate}

\begin{definition}
  Множество блоковых множеств является покрытием множества вершин.
\end{definition}
\begin{definition}
  Вершинный $(a,b)$-сепаратор - множество вершин графа, таких, что при их
  удалении $a$ и $b$ окажутся в различных компонентах.
\end{definition}

\begin{theorem}[Менгера]
  Наименьшее число вершин разделяющих 2 несмежные вершины a и b равно наибольшему числу попарно непересекающихся простых (a,b) - цепей графа.
\end{theorem}

\begin{theorem}[Уитни]
  Граф k-связный тогда и только тогда, когда любая пара его несовпадающих вершин соединена не менее чем k попарно непересекающимися цепями.
\end{theorem}
\begin{definition}
  Граф называется локально-связным если для каждой вершины v связен граф N(v).
\end{definition}


\chapter{Независимость и покрытие}
\begin{definition}
  Множество вершин называется независимым, если любые две вершины не смежны (индуцированный граф пуст).
\end{definition}
\begin{definition}
  $\alpha_0(G)$ - мощность наибольшего независимого множества вершин.
\end{definition}

\begin{theorem}
  Для любого графа справедливо неравенство
  $\alpha_0(G) >= \sum_{v \in VG}1 / (1 + \deg v)$
\end{theorem}
\begin{consequence}
  Для любого графа порядка n справедливо
  $\alpha_0(G) >= n / (1 + d) >= n / (1 + \Delta)$,
  d - среднее арифметическое степеней вершин.
\end{consequence}

\begin{theorem}
  Пусть G - связный граф порядка $n \geqslant 3$. $G$ не содержит $C_3$ и не
  является цепью либо циклом нечётной длины, тогда справедливо
  $\alpha_0(G) >= n / (\Delta(\Delta + 1)) + \sum_{v \in VG}1 / (1 + \deg v)$
\end{theorem}

\begin{theorem}
  Для любого графа G справедливо
  \begin{align*}
    \alpha_0(G) \leqslant p^0 + \min\set{p^-, p^+}
  \end{align*}
  \begin{itemize}
    \item $p^0$ - число нулевых собственных значений матрицы смежности графа.
    \item $p^-$ - число отрицательных собственных значений.
    \item $p^+$ - число положительных собственных значений.
  \end{itemize}
\end{theorem}

\begin{definition}
  Подмножество $V' \subset V(G)$ называется вершинным покрытием графа $G$, если
  его вершины покрывают все рёбра графа.
\end{definition}
Основная задача: поиск наименьшего покрытия (Vertex Cover).

\begin{definition}
  $\beta_0(G)$ - мощность наименьшего покрытия - число вершинного покрытия.
\end{definition}

Дополнение независимого множества в G = покрытие в G (и наоборот).

\begin{theorem}
  Множество P является наименьшим покрытием тогда и только тогда, когда
  дополнение P - наибольшее независимое множество. Таким образом,
  \begin{align*}
    \alpha_0(G) + \beta_0(G) = n.
  \end{align*}
\end{theorem}

\begin{definition}
  Подмножество $E' \subset E(G)$ называется независимым, если все его элементы
  попарно не смежные (паросочетание, matching).
\end{definition}

\begin{definition}
  Паросочетание называется совершенным, если  покрывает все вершины графа.
\end{definition}

\begin{definition}
  Подмножество рёбер называется рёберным покрытием графа, если оно покрывает все вершины графа.
\end{definition}

\begin{definition}
  $\alpha_1(G)$ - число рёберной независимости, число паросочетаний.
\end{definition}

\begin{definition}
  $\beta_1(G)$ - мощность наименьшего рёберного покрытия - число рёберного покрытия.
\end{definition}

\begin{theorem}[Галлаи]
  Пусть $G$ - граф без изолированнных вершин. Тогда верно равенство
  \begin{align*}
    \alpha_1(G) + \alpha_1(G) = n.
  \end{align*}
\end{theorem}

\begin{definition}
  $\forall A \subset V(G), N_G(A) = \cup_{v \in A}N(v) \ A$ - окружение множества А.
\end{definition}
\begin{theorem}[Холла]
  Для существования в двудольном графе $G(X, Y, E)$ паросочетания, покрывающего
  $X \Leftrightarrow (\forall A \subset X \Rightarrow |A| \leqslant |N_G(A)|)$
\end{theorem}
\begin{consequence}
  Любой непустой регулярный двудольный граф имеет совершенное паросочетание.
\end{consequence}
\begin{consequence}
  Регулярный двудольный граф ненулевой степени является рёберно непересекающимся
  объединением 1-факторов (1-фактор = совершенное паросочетание).
\end{consequence}

\begin{theorem}
  Для существования в двудольном графе $G$ паросочетания мощности $t$ необходимо
  и достаточно, чтобы выполнялось
  \begin{align*}
    \forall A \subset X, t \leqslant |X| \Rightarrow N(A) \geqslant |A| + t - |X|
  \end{align*}
\end{theorem}
\begin{theorem}
  Пусть $G(X,Y,E)$ - двудольный граф, $А \subset X$. Величину $\delta_0 = \max\delta(A) = \max(|A|-|N(A)|), A \neq \emptyset, \delta(A)=0, A = \emptyset)$  называют дефицитом графа G.
\end{theorem}

\begin{theorem}
  Для произвольного двудольного графа $G(X, Y, E)$ c непустыми долями справедливо:
  \begin{align*}
    \alpha_1(G) = |X| - \delta_0
  \end{align*}
\end{theorem}

\begin{theorem}
  Для произвольного двудольного графа G(X, Y, E)  c непустыми долями справедливо:
  \begin{align*}
    \beta_0(G) = \alpha_1(G)
  \end{align*}
\end{theorem}

\begin{theorem}[Татта]
  Для существования совершенного паросочетания в графе G необходимо и
  достаточно, чтобы для $\forall S \subset V(G)$ выполнялось $\rho(S) \leqslant |S|$,
  где $\rho(S)$ - число компонент нечетного порядка графа $G-S$.
\end{theorem}

\begin{theorem}[двуцветная теорема Рамсея]
  Для любого натурального p существует $n_0$, что при $n \geqslant n_0$ любой
  $n$-вершинный граф содержит или независимое множество мощности $p$ или клику
  размера $p$. Число $n_0$ в этом случае называют числом Рамсея для $p$.
\end{theorem}

Цепочка утверждений, показывающая полиномиальную разрешимость задачи о независимости в классе двудольных графов:
Теорема Холла -> теорема 14 -> определение дефицита -> теоремы 15,16.




\chapter{Планарность}
\begin{definition}
  Жорданова кривая - непрерывная спрямляемая линия, не имеющая самопересечений.
\end{definition}

\begin{definition}
  Множество D называется доминирующим, если любая vD смежна с D.
\end{definition}

\begin{theorem}[Жордана]
  Если есть замкнутая жорданова кривая, то она разбивает плоскость на две части.
  Незамкнутые кривые - не разбивают. Примечание: теорема неверна для тора.
\end{theorem}

\begin{definition}
  Пусть имеется произвольный непустой граф G. $L(G)$ - граф, в котором вершинами
  являются рёбра $G$, а рёбрами - наличие общей концевой вершины.
\end{definition}

\begin{definition}
  Граф Н называется реберным если G, что $H = L(G)$.
\end{definition}

\begin{definition}
  Множество $E’ \subset E(G)$называется разделяющим если $G \\ E'$ содержит
  большее число компонент, чем $G$. Минимальное разделяющее множество называется
  разрезом.
\end{definition}

\begin{stm}
  Всякий подграф планарного графа является планарным графом.
\end{stm}

\begin{stm}
  Граф планарен тогда и только тогда, когда все его связные компоненты планарны.
\end{stm}

\begin{definition}
  Гранью плоского графа называется максимальное множество точек плоскости,
  каждая пара которых может быть соединена жордановой кривой без пересечения
  ребер графа. Каждая точка плоскости принадлежит хотя бы одной грани.
\end{definition}

\begin{stm}
  Любую грань плоского графа можно сделать внешней в некоторой плоской укладке.
\end{stm}

\begin{stm}
  Возможно склеивание двух компонент связности по ребру или вершине без потери
  свойства планарности.
\end{stm}

\begin{stm}
  Для всякого плоского графа произвольная точка не лежащая на ребре принадлежит
  ровно 1 грани. Точка, принадлежащая ребру, но не являющаяся концевой вершиной
  этого ребра, принадлежит ровно 1 грани если ребро является мостом и 2 граням
  в противном случае.
\end{stm}

\begin{theorem}[Эйлера]
  Для каждого связного плоского графа верно равенство $n - m + f = 2$.
\end{theorem}

\begin{consequences}
  \item Граф имеет единственную укладку т. и т.т., когда он является треугольником.
  \item (Балинского). Остов n-мерного многогранника должен быть n-связен.
  \item Если данный $(n,m)$-граф связен и планарен, то верно неравенство
  $m <= 3n - 6$.
  \item Пусть имеется $(n,m)$-граф, у которого граница каждой грани является
  простым циклом длины $r$, связна и планарна. Тогда верно $m(r - 2)=r(n - 2)$
\end{consequences}

\begin{theorem}
  Плоский граф 2-связен т. и т. т., когда граница всякой его грани является простым циклом.
\end{theorem}

\begin{theorem}
  Связный граф планарен т. и т. т. когда каждый его блок планарен.
\end{theorem}

\begin{definition}
  Плоская триангуляция - граф, в котором каждая грань является (ограничена) треугольником.
\end{definition}

\begin{definition}
  Плоская триангуляция - $3$ -связный граф.
\end{definition}

\begin{theorem}
  Граф является максимальным плоским графом т. и т. т. когда он является плоской триангуляцией.
\end{theorem}

\begin{consequences}
  \item Каждый плоский граф является подграфом некоторой триангуляции.
  \item Для любого максимального планарного (n,m)-графа верно m = 3n - 6.
\end{consequences}

\begin{definition}
  G называется внешнепланарным если есть грань, к которой принадлежат все его вершины.
\end{definition}

\begin{stm}
  Если G - плоская триангуляция, n>=4 , то для |G| имеем (G)>=3.
\end{stm}

\begin{stm}
  Всякий планарный граф с n >= 4 имеет по крайней мере 4 вершины с deg(v) <= 5;
\end{stm}

\begin{definition}
  Два графа называют гомеоморфными, ели они могут быть получены из некоторого
  графа путем применения операций подразбиения его рёбер.
\end{definition}

\begin{theorem}[Понтрягина-Куратовского]
Граф планарен т. и т. т. когда он не содержит графов, гомеоморфных K5 и K3,3
\end{theorem}

\begin{theorem}[Вагнера]
  Граф планарен т. и т. т. когда в нём нет графов, стягиваемых к K5 и K3,3
\end{theorem}

\begin{definition}
  Геометрически двойственный граф $(G*)$. В каждой грани выберем по одной точке -
  вершины будущего графа. Для каждого ребра исходного графа соединим ребром
  вершины граней по разные его стороны (если ребро было мостом - будет петля).
\end{definition}

Свойства геометрически двойственного графа:
\begin{enumerate}
  \item $n^* = f, m^* = m, f^* = n$.
  \item Если $G$ является связным плоским графом, то $G^{**} G$
  \item Пусть $G$ является произвольным плоским графом. Подмножество рёбер
  образует в $G$ простой цикл тогда и только тогда, когда в $G^*$
  соответствующие рёбра образуют разрез.
  \item Пусть $G$ является произвольным связным графом. Если существует биекция,
  переводящая все простые циклы $G$ в разрезы $G^*$ и наоборот, то $G^*$ -
  абстрактно-двойственный.
\end{enumerate}

\begin{theorem}[Уитни]
  Граф планарен $\Leftrightarrow$ когда он имеет абстрактно-двойственный.
\end{theorem}

\begin{theorem}[Вагнера]
  Всякий планарный граф может быть уложен на плоскости так, что каждое его
  ребро будет прямолинейным отрезком.
\end{theorem}

Характеристики непланарности:
\begin{enumerate}
  \item Род графа $\gamma(G)$ - минимальный род поверхности, на которую граф
  может быть уложен
  \begin{align*}
    &\gamma(G) >= ](m-3n)/6 + 1[\\
    &\gamma(Kn) = ](n-3)(n-4)/12[
  \end{align*}
  \item Число скрещиваний $cr(G)$ - наименьшее число пересечений рёбер для
  укладки графа на плоскости (NP-трудная задача)
  \begin{align*}
    &cr(Kn) \leqslant \dfrac{1}{4}[n/2] [(n-1)/2][(n-2)/2] [(n-3)/2]\\
    &cr(K_{p,q}) = [p/2] [(p-1)/2][q/2] [(q-1)/2]
  \end{align*}
  \item Толщина графа $t(G)$ - наименьшее число его планарных подграфов,
  объединение которых даёт весь граф (NP-трудная задача)
  \begin{align*}
    &t(G) >=]m/(3n-6)[\\
    &t(Kn) >= [(n+7)/6],   = при n != 4 mod 6, n != 9
  \end{align*}
  \item Искажённость графа $sk(G)$ - наименьшее число рёбер, удаление которых
  сделает граф планарным (NP-трудная задача)
  \begin{align*}
    sk(K_n) = C_n^2 - 3n + 6, n \geqslant 3
  \end{align*}
\end{enumerate}

\chapter{Обходы в графах}
\begin{definition}
  Простой цикл в графе называется эйлеровым, если он содержит все рёбра графа.
  Цепь с аналогичными свойствами также называют йлеровой.
\end{definition}
\begin{definition}
  Набор рёбер на непересекающихся цепях покрывает граф если все рёбра являются
  рёбрами графа.
\end{definition}

\begin{theorem}[Эйлера]
  Связный граф является эйлеровым тогда и только тогда, когда степени всех его
  вершин чётные.
\end{theorem}

\begin{consequence}
  Пусть связный граф G содержит ровно $k$ вершин нечётной степени. Тогда число
  непересекающихся цепей, покрывающих граф, равно $k/2$.
\end{consequence}

\begin{theorem}[Коцига]
  Пусть $С$ и $С'$ - два различных эйлерова цикла графа $G$. Тогда существует
  такая последовательность $C=c_1,c_2,...,c_t=C'$ эйлеровых циклов в этом графе,
  что $c_j$ получается из $c_{j+1}$ изменением порядка обхода некоторого
  подцикла на обратный.
\end{theorem}

\begin{definition}
  Простой цикл называется гамильтоновым, если он проходит через каждую вершину
  графа.
\end{definition}

\begin{theorem}[Дирака]
  Если $n \geqslant 3$ и $\forall v \in V(G), deg v >= n/2$, то граф гамильтонов.
\end{theorem}

\begin{theorem}[Оре]
  Если $n \geqslant 3$ и $\forall v, u \in V(G), \deg v + \deg u \geqslant n$,
  то граф гамильтонов
\end{theorem}

\begin{theorem}[ХвАтала]
  Пусть $G$ имеет степенную последовательность $d_1 \leqslant d_2 \leqslant
  \ldots \leqslant d_n$. Граф $G$ гамильтонов, если $\forall k \in [1,n/2[$
  справедлива импликация $d_k \leqslant k \Rightarrow d_{n-k} \geqslant n-k$.
\end{theorem}

\begin{definition}
  Граф называется панциклическим если он содержит все циклы от $C_3$ до $C_n$.
\end{definition}

\begin{theorem}[Уитни]
  Плоская триангуляция без разделяющих треугольников (треугольников, которые не
  являются границей грани) является гамильтоновым графом.
\end{theorem}

\begin{theorem}[Татта]
  Всякий 4-связный планарный граф гамильтонов.
\end{theorem}

\begin{theorem}[Кляйшнера]
  Если ?? и $G$ - 2-связный, то $G^2$ гамильтонов.
\end{theorem}

\begin{theorem}
  Если граф G связный, локально-связный,  claw-free (, т. е. не содержит такого индуц. подграфа), то G - гамильтонов.
\end{theorem}

\begin{definition}
  Обхватом графа называется длина минимального простого цикла в этом графе.
\end{definition}

\begin{theorem}[Гринберга]
  Пусть $f_k$ - число k-граней плоского гамильтонового графа. Тогда для каждого
  $k \geqslant 3$ существуют $f'_k,f''_k$ - целые неотрицательные числа, т. ч.
  $f'_k + f''_k = f_k$  и
  \begin{align*}
    \sum_{k \geqslant 3}f'_k*(k-2) = \sum_{k \geqslant 3}f''_k*(k-2)
  \end{align*}
\end{theorem}

\begin{theorem}[Смита]
  Через всякое ребро кубического графа проходит чётное число гамильтоновых циклов.
\end{theorem}


\chapter{Раскраски}
\begin{definition}
  Раскраска - произвольная функция, переводящая множество вершин в множество цветов.
\end{definition}

\begin{definition}
  Правильная раскраска - смежные вершины имеют различные цвета.
\end{definition}

\begin{definition}
  Граф наз. k-раскрашиваемым если для него существует правильная раскраска в k цветов.
\end{definition}

\begin{definition}
  Хроматическое число графа - минимальное k для которого граф  к-раскрашиваем.
\end{definition}

\begin{definition}
  Карта - плоский граф без мостов.
\end{definition}

\begin{definition}
  Раскрака карты - отображение множества граней карты в множество цветов.
\end{definition}

\begin{definition}
  Хроматическое число поверхности - минимальное k для которого любая карта укладываемая на этой поверхности  к-раскрашиваема.
\end{definition}

Задача о 4 красках: Для любого плоского графа $\chi(G) \leqslant 4$.

\begin{theorem}[Хилла]
  Всякий планарный граф является  5-раскрашиваемым.
\end{theorem}

\begin{theorem}[Крола]
  G – плоский 3-раскрашиваемый $\Leftrightarrow$ он является подграфом чётной триангуляции (все степени вершин чётные).
\end{theorem}

\begin{consequence}
  Граф G можно раскрасить в 3 цвета, если в нём не более 4 треугольников.
\end{consequence}

\begin{theorem}[Рингеля-Янгса]
  Пусть $S_p$ - сфера с $р$ ручками (поверхность рода $р$). Тогда:
  \begin{align*}
    \chi(S_p) = \left[\dfrac{7 + \sqrt{1+48p}}{2}\right], p > 0
  \end{align*}
\end{theorem}

Гипотеза Хадвигера: Любой связный n-хроматический граф стягивается к $K_n$.
Доказано для n = 1, 2, 3, 5.

\begin{theorem}[Брукса]
  Пусть G - связный, $G \neq K_n$, $\Delta(G) \geqslant 3$. Тогда  $\chi(G) \leqslant \Delta(G)$.
\end{theorem}

\begin{definition}
  Хроматическая функция (хроматический полином) - число попарно-различных
  правильных раскрасок в t цветов.
\end{definition}

\begin{align*}
  &f_0(n, t) = t^n,\\
  &f(K_n, t) = { t(t-1)(t-2)...(t-n+1), t >= n \text{ и } 0 \text{ при } t < n }
\end{align*}

\begin{stm}
  Если $G_1, G_2, ..., G_k$– связные компоненты $G$, тогда
  \begin{align*}
    f(G, t) = \prod_{i = 1}^kf(G_i, t)
  \end{align*}
\end{stm}

\begin{stm}
  Пусть $G = G_1 \cup G_2$ и графы $G_1, G_2$ имеют в точности одну общую
  вершину, тогда:
  \begin{align*}
    f(G, t)= \dfrac{f(G_1, t) \cdot f(G_2, t)}{t}
  \end{align*}
\end{stm}

\begin{stm}
  Пусть $u$ и $v$ - две несмежные вершины $G$, и пусть $G_1 = G + uv$, $G_2$ –
  граф, полученный стягиванием вершин $u$ и $v$. Тогда:
  \begin{align*}
    f(G, t)= f(G_1, t) + f(G_2, t)
  \end{align*}
\end{stm}

\begin{theorem}
  Хроматический полином $(n,m)$-графа $G$, имеющего в точности $k$ связных
  компонент:
  \begin{align*}
    f(G,t) = t^n -mt^{n-1} + a_{n-2}t^{n-2} - a_{n-3}t^{n-3} + \ldots
    + (-1)^{n-k}t^k,
  \end{align*}
  где $a_i$ – целые неотрицательные числа
\end{theorem}

\begin{theorem}
  Граф $G, |G| = n$, дерево $\Leftrightarrow f(G,t)=t(t - 1)^{n - 1}$
\end{theorem}

\begin{definition}
  Совершенный граф - если $\chi(H) = \phi(H)$для любого индуцированного
  подграфа $H$ ($\phi(G)$ - размер максимальной клики (плотность графа) ).
\end{definition}

\begin{theorem}[Поль Сеймур + Берт, гипотеза Бержа]
  Граф совершенный $\Leftrightarrow$ ни $G,$ ни его дополнение не содержат
  индуцированных циклов нечётной длины.
\end{theorem}

\begin{consequences}
  \item $G$ – совершенный $\Leftrightarrow$ дополнение $G$ – совершенный.
  \item Двудольные графы и их дополнения совершенны.
  \item Триангулированные графы и их дополнения совершенны (граф
  триангулированный, если ни один его порожденный подграф не является
  простым циклом длины больше трёх).
\end{consequences}

\begin{definition}
  Рёберная раскраска $G$ = вершинная раскраска $L(G)$
\end{definition}

\begin{definition}
  Хроматический индекс $\chi'(G)$- минимальное число, в которое правильно
  раскрашиваемы рёбра.
\end{definition}
\begin{theorem}[Визинга]
  $\Delta(G) \leqslant \chi'(G) \leqslant \Delta(G) + 1$.
\end{theorem}

\chapter{Ориентированные графы}
\begin{definition}
  Простой цикл -> контур. Простая цепь -> путь.
\end{definition}

\begin{definition}
   Орграф называется сильносвязным когда любая вершина достижима из любой другой.
\end{definition}

\begin{definition}
  Орграф называется одностороннесвязанный когда для любой пары вершин одна из
  них достижима из другой.
\end{definition}

\begin{stm}
  Орграф является сильным тогда и только тогда, когда в нём есть остовный
  циклический маршрут.
\end{stm}

\begin{stm}
  Орграф является односторонним тогда и только тогда, когда в нём есть
  остовный маршрут.
\end{stm}

\begin{definition}
  Сильной компонентой называется любой максимальный сильный подграф.
\end{definition}

\begin{definition}
  Конденсация графа - сливаем вершины каждой сильной компоненты в одну.
\end{definition}

\begin{stm}
  Конденсация любого орграфа не имеет контуров.
\end{stm}

\begin{definition}
  Полустепени исхода и захода
\end{definition}

\begin{definition}
  Орграф называется турниром если его основание является графом $К_n$.
\end{definition}

\begin{theorem}
  Связный ориентированный граф эйлеров тогда и только тогда, когда
  \begin{align*}
    \forall v \in V(G) \Rightarrow d^+(v) = d^-(v).
  \end{align*}
\end{theorem}

\begin{theorem}[Мейнела]
  Пусть $G$ - сильный, $n > 1$. Если для любой его пары $u != v$ несмежных
  вершин выполняется неравенство $\deg u + \deg v >= 2n - 1$ то орграф гамильтонов.
\end{theorem}

\begin{theorem}
  Для любого орграфа $G$ справедливы неравенство $l(G)<= \alpha_0(G)$, где
  $l(G)$ - минимальное число путей, на которое можно разбить $G$, а
  $\alpha_0(G)$- мощность наибольшего независимого множества вершин.
\end{theorem}

\begin{definition}
  Орграф называется транзитивным если из того что $(x,y)$ и $(y, z)$ принадлежат
  графу автоматически следует, что $(x,z)$ также принадлежит.
\end{definition}

\begin{consequences}
  \item Если орграф транзитивный, то $l(G) = \alpha_0(G)$.
  \item В каждом турнире существует гамильтонов путь.
\end{consequences}

\begin{theorem}[Галлан]
  Пусть $k$ - максимальная длина пути в орграфе. Тогда $\chi(G) \leqslant k+1$.
\end{theorem}

\chapter{Степенные последовательности}
\begin{definition}
  Степенная последовательность - последовательность степеней вершин
  \begin{align*}
    d_1 \geqslant d_2 \geqslant \ldots \geqslant d_n
  \end{align*}
\end{definition}

\begin{definition}
  Последовательность чисел, для которой существует соответствующий граф
  называется графической.
\end{definition}

\begin{definition}
  Переключение: Пусть $ab,cd \in E(G)$, $ac,bd \notin E(G)$. Тогда
  говорят, что $G$ допускает переключение
  (удаляется первая пара ребер и добавляется вторая).
  \begin{align*}
    ((a, b), (c, d)) -> ((a, c), (b, d))
  \end{align*}
\end{definition}

\begin{lemma}
  Пусть $VG = \set{1, 2, … n}, \deg i = d_i, d_i >= d_{i + 1}$. Тогда для
  любого $i$ существует такая последовательность переключений, что в новом графе
  окружение вершины $i$ совпадает с $d_i$ вершинами наибольших степеней.
\end{lemma}

\begin{theorem}
  Любая реализация графической последовательности получается из любой другой её
  реализации путём применения подходящей цепочки переключений.
\end{theorem}

\begin{definition}
  Униграф - граф, степенная последовательности которого допускает
  единственную реализацию.
\end{definition}

\begin{definition}
  Операция получения производной последовательности:
  \begin{enumerate}
    \item Удаляем $d_i$.
    \item Из первых $d_i$ элементов вычитаем 1.
  \end{enumerate}
\end{definition}

\begin{theorem}[Критерий Гавела]
  $d$ - правильная $n$-последовательность. Если для какого-то индекса $i$
  производная последовательность $d^i$ - графическая, то исходная
  последовательность - тоже графическая. Если $d$ - графическая, то любая
  последовательность $d^i$ - графическая.
\end{theorem}

\begin{theorem}[Эрдёш]
  Правильная n-последовательность является графической $\Leftrightarrow$ для
  каждого $1 \leqslant k \leqslant n-1$ выполняется неравенство
  \begin{align*}
    \sum_{i = 1}^n d_i \leqslant k(k - 1) +  \sum_{i = k + 1}^n \min(k, di)
  \end{align*}
\end{theorem}

\begin{theorem}
  Правильная графическая n-последовательность d имеет связную реализацию
  $\Leftrightarrow d_n \geqslant 0$ и выполняется неравенство:
  \begin{align*}
    \sum_{i = 1}^n d_i \geqslant 2(n - 1).
  \end{align*}
\end{theorem}

\begin{theorem}
  Последовательность реализуется деревом $\Leftrightarrow$
  \begin{align*}
    \sum_{i = 1}^n d_i = 2(n - 1)
  \end{align*}
\end{theorem}

\begin{theorem}
  Пара векторов $c = (c_1, ..) и d = (d_1, ..)$ является графической для орграфа
  ($c$ - степени исхода) $\Leftrightarrow$ выполняется:
  \begin{enumerate}
    \item $(c_1 + n - 1, c_2 + n - 1, … c_n + n - 1, d_1, … d_n)$ - графическая
    для неориентированного графа.
    \item $\sum_{i = 1^n}c_i = \sum_{i = 1}^n d_i$
  \end{enumerate}
\end{theorem}

\chapter{Гиперграфы}
 
\chapter{Теория сложности}


\end{document}
